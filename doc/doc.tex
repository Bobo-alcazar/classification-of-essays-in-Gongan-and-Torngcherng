\documentclass[12pt, a4paper, oneside]{ctexart}
\usepackage{amsmath, graphicx, geometry, hyperref, listings, multirow}


% 导言区

\title{基于scikit-learn库的文本分类任务测试}
\author{叶蓁 211506011}
\date{}

\setCJKmainfont{I.Ming-8.00.ttf}
\setCJKsansfont{LXGWClearGothic-Regular.ttf}
\setCJKmonofont{SourceHanMono.ttc}

%\geometry{left=1.5cm, right=1.5cm, top=3cm, bottom=3cm}

\begin{document}

\maketitle

本任务代码已全部上传至\href{https://github.com/Bobo-alcazar/classification-of-essays-in-Gongan-and-Torngcherng}{GitHub},故而不再单独展示代码。部分文件由于体积过大未上传,这包含jiayan库的支持模型、LTP库的支持模型,及本任务所训练的文本分类模型。前两者可根据其文档指示下载到,执行根目录下\href{https://github.com/Bobo-alcazar/classification-of-essays-in-Gongan-and-Torngcherng/blob/main/classifier.py}{\verb|classifier.py}即会在\href{https://github.com/Bobo-alcazar/classification-of-essays-in-Gongan-and-Torngcherng/tree/main/module}{\verb|./module}目录下生成模型。

\section{任务简介}

本任务目的是进行一次基于传统机器学习的文本分类任务实践。

现今机器学习已经有大量很成熟的工具,不需要我们从头开始设计。总的来说,有低代码可视化平台和编程工具包两类工具可供我们使用,为了更加自由地处理数据和更加便捷地测试和输出测试结果,我们选择使用编程语言工具包来完成。具体来说,我们选择使用python语言的scikit-learn库来完成,因为scikit-learn库是当前最为常用的机器学习工具包,且提供的算法丰富,可供我们进行大量比较,API结构清晰,便于使用,还提供了便捷的测试结果统计和评估函数。

本任务选取的分类对象是明末公安派与清初桐城派的散文。在文学史上,桐城派很大程度是为反对公安派而产生的,两派散文有较大差异,应当能够被简单的基于机器学习的文本分类模型区分。两派的文章差异体现在用词、句法、篇章结构、思想内涵等多个方面,后两个部分一般来说由篇章分析和主题提取两类任务负责。本任务仅基于用词特点,建立词袋模型(Bag-of-words model)进行训练。

总的流程是,先对两派散文分别选取一定的范围,并在网络上爬取,存储下来。需要注意的是,不同网络资源的结构化程度不同,结构化较强的,已以篇为单位整理好,并施句读;结构化较弱的,可能不仅无句读,且连篇章界限亦无显性标识。对于后者,需要根据文本规律划分篇章。在可能的情况下,应尽量选取结构化程度更高的资料,以避免预处理中不必要的工作。

接下来,对文本进行预处理。具体来说,对于无句读的文档进行断句。尔后对每篇文档进行分词处理,将每篇文档表示为一个字符串序列。在此之后,根据字符串序列建立独热(one-hot)表示或TF-IDF加权表示,将每篇文档表示为一个向量。

将向量化的文档分别以互信息筛选和方差筛选两种方法降维,并以一定比例划分为训练集和测试集后,分别采取支持向量机(SVM)、对数几率回归(logistic regression)、朴素贝叶斯(naive Bayes)、K近邻(K-nearest neighbors)、决策树(decision tree)、随机森林(random forest)六种算法根据训练集进行建模,并应用于测试集。记录下测试集的预测结果与实际结果,计算每种情况的F值与AUC值,分析对于此任务,各算法的优劣。

\section{语料库制作}

任务的第一步是爬取网络资源制作语料库。对于公安派和桐城派,我们分别选取其最具代表性的公安三袁的作品集,袁宏道\href{https://ctext.org/wiki.pl?if=gb&res=142162&remap=gb}{《袁中郎全集》}、袁中道\href{https://zh.wikisource.org/wiki/珂雪齋集}{《珂雪斋集》}、袁宗道\href{https://zh.wikisource.org/zh-hant/白蘇齋類集}{《白苏斋类集》}和桐城三祖的作品集方苞\href{https://zh.wikisource.org/wiki/方望溪先生全集_(四部叢刊本)}{《方望溪先生全集》}、刘大櫆\href{https://ctext.org/wiki.pl?if=gb&res=110875}{《海峰文集》}、姚鼐\href{https://zh.wikisource.org/wiki/惜抱軒文集}{《惜抱轩文集》},它们分别可以在维基文库和中国哲学书电子化计划两网站上看到,且每部集子都配有带链接的目录页,便于爬取。维基媒体的所有网站都是不设反爬措施的(实际上维基媒体鼓励合规的自由利用网站数据),中国哲学书电子化计划网站设有反扒措施,但并不严密,且主要是针对爬取网站公开书影的,对文本信息页面爬取管得不严,综上,我们不专门设计避免反扒措施的流程,不过伪装浏览器的请求头还是必要的。

此外,本任务仅是对散文流派的文本分类,其他文体不仅在用词上与散文有较大差异,在文学上也不被纳入桐城派、公安派的分类之中。以上作品集有的并非是散文类集性质,其中包含的其他文体因而是需要排除的,如《袁中郎全集》卷一至卷九、《珂雪斋集》卷一至卷八、《白苏斋类集》卷一至卷六。还有一些目录、书序一类章节,其中不包含我们所需语料,也是需要排除的。这些要排除的内容被硬编码进了脚本。

《珂雪斋集》、《白苏斋类集》和《惜抱轩文集》的结构化程度都比较高,篇目标题皆在\lstinline[language=html]!<h2>!或\lstinline[language=html]!<h3>!标签中,而文章内容则皆在标题后的\lstinline[language=html]!<p>!标签中,直接顺次读取即可。《海峰文集》基本上也符合这样的格式,只文章被放在了\lstinline[language=html]!<table>!中,不过不是什么大问题。比较麻烦的是《方望溪先生全集》和《袁中郎全集》,前者是对古籍扫描后得到的文本,没有经过任何其他整理,格式也较乱;后者是照古籍录入的,经过了一定校勘,格式上与古籍保持完全一致,未经结构化。观察可知,后者凡标题行皆空两个全角空格,可据此判断标题;前者则无此特点,我们只能根据一行的长度,断定较短的是标题,较长的是内容,经检验,这种办法基本能正确地把每篇文章区分开。这也就足够了,毕竟本任务计划使用的是词袋法,对于篇章分割的鲁棒性较高,可以容许少量错误划分而不影响结果。

爬取后,每个文档被存储为一个三元列表,分别存放派别、作者和文档文本信息,以JSON格式存储在\href{https://github.com/Bobo-alcazar/classification-of-essays-in-Gongan-and-Torngcherng/tree/main/corpus/raw}{\verb!./corpus/raw}下。本部分的代码见于根目录下的\href{https://github.com/Bobo-alcazar/classification-of-essays-in-Gongan-and-Torngcherng/blob/main/crawler.py}{\verb!crawler.py}文件。

\section{文本预处理}

这里的文本预处理任务,主要包括断句和分词两项。

《方望溪先生全集》、《海峰文集》、《袁中郎全集》无句读,因此需要先进行断句任务;另三者已有标点,需在分词后去除其中标点。实际上,桐城派非常重视“文气”概念,这很大程度即由句子节奏决定,因而逗号和句号所反映的句子长度和小句长度是重要的文本特征。但由于我们的语料并非都有句读,而目前又无公开的区分句号与逗号的断句工具,因而我们统一不考虑这方面信息,仅使用单纯的词袋模型。(我怀疑加入句法结构信息作为特征亦能大幅度提高分类效果,不过本任务暂不考虑)

大部分NLP工具包仅能接受单个句子执行分词任务,仅有stanza、LTP、HanLP、jiayan几个工具包可进行分句任务。HanLP和jiayan对文言的处理能力很强,是经过验证的。但HanLP的分句系连接服务器调用API完成,而我们需要处理的文本量极大,HanLP又对限定时间内API调用次数有限制,我们首先不考虑使用。stanza的中文(mandarin)模型即使对现代汉语的处理能力都较差,更不谈对文言的处理能力。不过,stanza专门针对文言训练了模型,不过模型体积大、速度慢,效果还未见得有明显优势。LTP的分句完全是基于标点符号的,无法承担句读任务。最终我们决定由jiayan来完成此任务。对于分词任务,亦决定由jiayan来完成。jiayan的算法虽然无stanza强大,但所使用的语料库远大于stanza的文言模型训练数据集。

我们手动存储了常用标点符号列表,在分词后去除。一种保险的办法是根据Unicode字符编码的汉字区(即CJKV Unified Ideographs、CJKV Extension A、CJKV Extension B、
CJKV Extension C、CJKV Extension D、CJKV Extension E、CJKV Extension F、CJKV Extension G、CJKV Extension H、CJKV Compatibility Ideographs、CJKV Compatibility Ideographs Supplement、U+3007)范围进行筛选,即去除汉字区以外的所有字符,不过那样会比较麻烦。考虑到有极少量标点未去除并不明显结果,我们未采用这种办法。严格来说,为了防止不同数位化习惯造成的问题,遇到CJKV Compatibility Ideographs(U+F900–U+FAFF)与CJKV Compatibility Ideographs Supplement(U+2F800–U+2FA1)中字符还应根据相关文件统合进其他九个区域中,此外,CJKV Unified Ideographs、CJKV Extension A、CJKV Extension B中的重出字也应统合(\href{https://glyphwiki.org/wiki/Group:原規格分離}{glyph wiki}有作不完整的整理工作,但目前似乎并无公开完整数据,这个工作实际上无法轻易地完成)。实际上,如果不同网站普遍采取不同习惯录入,是确实会严重干扰文本处理的,但基于㈠目前无完整公开数据,较难完成该任务;㈡语料规范性本来就差,不差这一项,而词袋模型鲁棒性较强;㈢我们假设用字习惯和语料来源网站没有强相关性;㈣即使有强相关性,也主要干扰的是文本聚类,对文本分类的干扰较小,以上四个理由,而不作这方面工作。

经过以上处理后,将每个文档表示为一个字符串序列,每个元素为一个词,并将作者、流派及文档表示存储起来,每个作者的文章存为一个JSON文件,保存在\href{https://github.com/Bobo-alcazar/classification-of-essays-in-Gongan-and-Torngcherng/tree/main/corpus}{\verb!./corpus}目录下。

本部分代码见于\href{https://github.com/Bobo-alcazar/classification-of-essays-in-Gongan-and-Torngcherng/blob/main/preprocess/segmentation.py}{\verb!./preprocess/segmentation.py}文件。

\section{文本分类模型训练}

上一步骤经过预处理,已经得到了结构化的文本数据。完整地来说,本步骤包含六个部分,分别是㈠去停用词;㈡分离训练集与测试集;㈢制作文档向量表示;㈣特征筛选;㈤模型训练;㈥模型测试。对于停用词,我们统计了文档中出现频率最高的数十个词汇,去除了其中的实词,制作成停用词表。(高频词统计见\href{https://github.com/Bobo-alcazar/classification-of-essays-in-Gongan-and-Torngcherng/blob/main/frequency.py}{\verb!frequency.py},停用词表见\href{https://github.com/Bobo-alcazar/classification-of-essays-in-Gongan-and-Torngcherng/blob/main/stopWords.json}{\verb!stopWords.json})。我们选取的停用词不多,这是考虑到,桐城派与公安派散文的用词风格差异很大程度是依赖于高频词差异的,而低频词反而和具体主题有关,而与风格关系不大。因此我们需更加谨慎地筛选停用词。训练集与测试集按三七开划分。文档向量表示有独热表示的词袋模型和TFIDF表示的加权词袋模型。特征筛选我们采取互信息筛选和方差筛选两种方式。之所以不考虑不加筛选的训练,是由于原文本向量维数过多,不加筛选的训练耗时极长,无法承担。模型训练分别采取支持向量机(SVM)、对数几率回归(logistic regression)、朴素贝叶斯(naive Bayes)、K近邻(K-nearest neighbors)、决策树(decision tree)、随机森林(random forest)六种算法。在根据训练数据集训练好模型后,即将分类模型应用于测试数据集,并计算F值及AUC值。

以上代码见于\href{https://github.com/Bobo-alcazar/classification-of-essays-in-Gongan-and-Torngcherng/blob/main/classifier.py}{\verb!classifier.py}。

\section{文本分类效果分析}

采取各种不同算法分类所测得的F值及AUC值原始输出见于\href{https://github.com/Bobo-alcazar/classification-of-essays-in-Gongan-and-Torngcherng/blob/main/Logout.txt}{\verb!Logout.txt},其中包含查准率、查全率及其宏平均值、加权平均值,以及AUC值。我们主要采取AUC值评估模型分类效果,下表是各算法中的AUC值。

\begin{table}[!htb]
    \centering
    \resizebox{\linewidth}{!}{
    \begin{tabular}{|c|c|c|c|c|c|c|c|c|}
        \hline
        ~ & ~ & \multicolumn{2}{c|}{词袋} & \multicolumn{2}{c|}{TFIDF} \\ \hline
        ~ & ~ & 不去停用词 & 去停用词 & 不去停用词 & 去停用词 \\ \hline
        \multirow{6}*{互信息筛选} & 支持向量机 & 0.81303097 & 0.81303097 & 0.771935466 & 0.771935466 \\ \cline{2-6}
        ~ & 对数几率回归 & 0.824724996 & 0.824724996 & 0.828668133 & 0.828668133 \\ \cline{2-6}
        ~ & 朴素贝叶斯 & 0.78149151 & 0.78149151 & 0.772866249 & 0.772866249 \\ \cline{2-6}
        ~ & K近邻 & 0.749145371 & 0.749145371 & 0.780718678 & 0.780718678 \\ \cline{2-6}
        ~ & 决策树 & 0.776634512 & 0.778682236 & 0.779849946 & 0.780289953 \\ \cline{2-6}
        ~ & 随机森林 & 0.781153043 & 0.781153043 & 0.781299712 & 0.781299712 \\ \cline{2-6}
        \multirow{6}*{方差筛选} & 支持向量机 & 0.893146048 & 0.893146048 & 0.902493372 & 0.902493372 \\ \hline
        ~ & 对数几率回归 & 0.927348113 & 0.927348113 & 0.899717944 & 0.902493372 \\ \cline{2-6}
        ~ & 朴素贝叶斯 & 0.937581091 & 0.937581091 & 0.852225419 & 0.852225419 \\ \cline{2-6}
        ~ & K近邻 & 0.523805494 & 0.523805494 & 0.559068088 & 0.559068088 \\ \cline{2-6}
        ~ & 决策树 & 0.764382016 & 0.75517008 & 0.828826085 & 0.825170644 \\ \cline{2-6}
        ~ & 随机森林 & 0.823698313 & 0.823698313 & 0.826327072 & 0.826327072 \\ \hline
    \end{tabular}
    }
\end{table}

从表中可以看出,才用方差筛选的方法表示的文档向量所训练的分类模型,效果普遍要远好于互信息筛选的效果,无论是单个比较还是整体来看都是如此。唯一的例外是,方差筛选后采用K近邻算法训练模型,效果尤其差,接近随机,完全不能使用。此外,才用朴素词袋法表示并加以方差筛选,以决策树算法训练的模型效果也较差,不如互信息筛选的效果好。

总体来看,同等条件下,不加权的词袋法表示与加权的TFIDF表示效果差异不相上下,差别并不明显。由于我们去除的停用词非常少,大部分数据中,是否去停用词得到的AUC值都是差不多的,但不去停用词效果要略好于去停用词的情况,这是符合预期的。

在控制文档表示和特征筛选条件的情况下,K近邻算法效果最差,对此任务不应使用;决策树算法分类效果也较差,但其在才用TFIDF表示和方差筛选时效果较好。随机森林算法的表现较平庸。对数几率回归算法和支持向量机算法整体表现最好,且尤其是在采取方差筛选特征的情况下。整体表现效果较好的,其次是朴素贝叶斯算法。此外,朴素贝叶斯算法在搭配不加权的词袋法表示且才用方差筛选特征时效果是所有组合中最好的,AUC值高于0.93。

\section{总结}

对于文言散文文体风格的分类,不去停用词效果可能好于去停用词的。TFIDF表示虽然携带了比词袋表示更多的信息,但分类效果未见得比词袋法好。方差筛选思路较为直观,但效果可能也很好。

总之,对于文本分类任务,应尽量尝试更多的表示和算法组合,找到最优的方法。






\end{document}