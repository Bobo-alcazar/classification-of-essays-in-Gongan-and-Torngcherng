\documentclass[12pt, a4paper, oneside]{ctexart}
\usepackage{amsmath, graphicx, geometry, hyperref, listings}


% 导言区

\title{基于scikit-learn库的文本分类任务测试}
\author{叶蓁 211506011}
\date{}

\setCJKVmainfont{I.Ming-8.00.ttf}
\setCJKVsansfont{LXGWClearGothic-Regular.ttf}
\setCJKVmonofont{SourceHanMono.ttc}

%\geometry{left=1.5cm, right=1.5cm, top=3cm, bottom=3cm}

\begin{document}

\maketitle

本任务代码已全部上传至\href{https://github.com/Bobo-alcazar/classification-of-essays-in-Gongan-and-Torngcherng}{GitHub},部分文件由于体积过大未上传,这包含jiayan库的支持模型、LTP库的支持模型,及本任务所训练的文本分类模型。前两者可根据其文档指示下载到,执行根目录下\verb!classifier.py!即会在\verb!./module!目录下生成模型。

\section{任务简介}

本任务目的是进行一次基于传统机器学习的文本分类任务实践。

现今机器学习已经有大量很成熟的工具,不需要我们从头开始设计。总的来说,有低代码可视化平台和编程工具包两类工具可供我们使用,为了更加自由地处理数据和更加便捷地测试和输出测试结果,我们选择使用编程语言工具包来完成。具体来说,我们选择使用python语言的scikit-learn库来完成,因为scikit-learn库是当前最为常用的机器学习工具包,且提供的算法丰富,可供我们进行大量比较,API结构清晰,便于使用,还提供了便捷的测试结果统计和评估函数。

本任务选取的分类对象是明末公安派与清初桐城派的散文。在文学史上,桐城派很大程度是为反对公安派而产生的,两派散文有较大差异,应当能够被简单的基于机器学习的文本分类模型区分。两派的文章差异体现在用词、句法、篇章结构、思想内涵等多个方面,后两个部分一般来说由篇章分析和主题提取两类任务负责。本任务仅基于用词特点,建立词袋模型(Bag-of-words model)进行训练。

总的流程是,先对两派散文分别选取一定的范围,并在网络上爬取,存储下来。需要注意的是,不同网络资源的结构化程度不同,结构化较强的,已以篇为单位整理好,并施句读;结构化较弱的,可能不仅无句读,且连篇章界限亦无显性标识。对于后者,需要根据文本规律划分篇章。在可能的情况下,应尽量选取结构化程度更高的资料,以避免预处理中不必要的工作。

接下来,对文本进行预处理。具体来说,对于无句读的文档进行断句。尔后对每篇文档进行分词处理,将每篇文档表示为一个字符串序列。在此之后,根据字符串序列建立独热(one-hot)表示或TF-IDF加权表示,将每篇文档表示为一个向量。

将向量化的文档分别以互信息筛选和方差筛选两种方法降维,并以一定比例划分为训练集和测试集后,分别采取支持向量机(SVM)、对数几率回归(logistic regression)、朴素贝叶斯(naive Bayes)、K近邻(K-nearest neighbors)、决策树(decision tree)、随机森林(random forest)六种算法根据训练集进行建模,并应用于测试集。记录下测试集的预测结果与实际结果,计算每种情况的F值与AUC值,分析对于此任务,各算法的优劣。

\section{语料库制作}

任务的第一步是爬取网络资源制作语料库。对于公安派和桐城派,我们分别选取其最具代表性的公安三袁的作品集,袁宏道\href{https://ctext.org/wiki.pl?if=gb&res=142162&remap=gb}{《袁中郎全集》}、袁中道\href{https://zh.wikisource.org/wiki/珂雪齋集}{《珂雪斋集》}、袁宗道\href{https://zh.wikisource.org/zh-hant/白蘇齋類集}{《白苏斋类集》}和桐城三祖的作品集方苞\href{https://zh.wikisource.org/wiki/方望溪先生全集_(四部叢刊本)}{《方望溪先生全集》}、刘大櫆\href{https://ctext.org/wiki.pl?if=gb&res=110875}{《海峰文集》}、姚鼐\href{https://zh.wikisource.org/wiki/惜抱軒文集}{《惜抱轩文集》},它们分别可以在维基文库和中国哲学书电子化计划两网站上看到,且每部集子都配有带链接的目录页,便于爬取。维基媒体的所有网站都是不设反爬措施的(实际上维基媒体鼓励合规的自由利用网站数据),中国哲学书电子化计划网站设有反扒措施,但并不严密,且主要是针对爬取网站公开书影的,对文本信息页面爬取管得不严,综上,我们不专门设计避免反扒措施的流程,不过伪装浏览器的请求头还是必要的。

此外,本任务仅是对散文流派的文本分类,其他文体不仅在用词上与散文有较大差异,在文学上也不被纳入桐城派、公安派的分类之中。以上作品集有的并非是散文类集性质,其中包含的其他文体因而是需要排除的,如《袁中郎全集》卷一至卷九、《珂雪斋集》卷一至卷八、《白苏斋类集》卷一至卷六。还有一些目录、书序一类章节,其中不包含我们所需语料,也是需要排除的。这些要排除的内容被硬编码进了脚本。

《珂雪斋集》、《白苏斋类集》和《惜抱轩文集》的结构化程度都比较高,篇目标题皆在\lstinline[language=html]!<h2>!或\lstinline[language=html]!<h3>!标签中,而文章内容则皆在标题后的\lstinline[language=html]!<p>!标签中,直接顺次读取即可。《海峰文集》基本上也符合这样的格式,只文章被放在了\lstinline[language=html]!<table>!中,不过不是什么大问题。比较麻烦的是《方望溪先生全集》和《袁中郎全集》,前者是对古籍扫描后得到的文本,没有经过任何其他整理,格式也较乱;后者是照古籍录入的,经过了一定校勘,格式上与古籍保持完全一致,未经结构化。观察可知,后者凡标题行皆空两个全角空格,可据此判断标题;前者则无此特点,我们只能根据一行的长度,断定较短的是标题,较长的是内容,经检验,这种办法基本能正确地把每篇文章区分开。这也就足够了,毕竟本任务计划使用的是词袋法,对于篇章分割的鲁棒性较高,可以容许少量错误划分而不影响结果。

爬取后,每个文档被存储为一个三元列表,分别存放派别、作者和文档文本信息,以JSON格式存储在\verb!./corpus/raw!下。本部分的代码见于根目录下的\verb!crawler.py!文件。

\section{文本预处理}

这里的文本预处理任务,主要包括断句和分词两项。

《方望溪先生全集》、《海峰文集》、《袁中郎全集》无句读,因此需要先进行断句任务;另三者已有标点,需在分词后去除其中标点。实际上,桐城派非常重视“文气”概念,这很大程度即由句子节奏决定,因而逗号和句号所反映的句子长度和小句长度是重要的文本特征。但由于我们的语料并非都有句读,而目前又无公开的区分句号与逗号的断句工具,因而我们统一不考虑这方面信息,仅使用单纯的词袋模型。(我怀疑加入句法结构信息作为特征亦能大幅度提高分类效果,不过本任务暂不考虑)

大部分NLP工具包仅能接受单个句子执行分词任务,仅有stanza、LTP、HanLP、jiayan几个工具包可进行分句任务。HanLP和jiayan对文言的处理能力很强,是经过验证的。但HanLP的分句系连接服务器调用API完成,而我们需要处理的文本量极大,HanLP又对限定时间内API调用次数有限制,我们首先不考虑使用。stanza的中文(mandarin)模型即使对现代汉语的处理能力都较差,更不谈对文言的处理能力。不过,stanza专门针对文言训练了模型,不过模型体积大、速度慢,效果还未见得有明显优势。LTP的分句完全是基于标点符号的,无法承担句读任务。最终我们决定由jiayan来完成此任务。对于分词任务,亦决定由jiayan来完成。jiayan的算法虽然无stanza强大,但所使用的语料库远大于stanza的文言模型训练数据集。

我们手动存储了常用标点符号列表,在分词后去除。一种保险的办法是根据Unicode字符编码的汉字区(即CJKV Unified Ideographs、CJKV Extension A、CJKV Extension B、
CJKV Extension C、CJKV Extension D、CJKV Extension E、CJKV Extension F、CJKV Extension G、CJKV Extension H、CJKV Compatibility Ideographs、CJKV Compatibility Ideographs Supplement、U+3007)范围进行筛选,即去除汉字区以外的所有字符,不过那样会比较麻烦。考虑到有极少量标点未去除并不明显结果,我们未采用这种办法。严格来说,为了防止不同数位化习惯造成的问题,遇到CJKV Compatibility Ideographs(U+F900–U+FAFF)与CJKV Compatibility Ideographs Supplement(U+2F800–U+2FA1)中字符还应根据相关文件统合进其他九个区域中,此外,CJKV Unified Ideographs、CJKV Extension A、CJKV Extension B中的重出字也应统合(\href{https://glyphwiki.org/wiki/Group:原規格分離}{glyph wiki}有作不完整的整理工作,但目前似乎并无公开完整数据,这个工作实际上无法轻易地完成)。实际上,如果不同网站普遍采取不同习惯录入,是确实会严重干扰文本处理的,但基于㈠目前无完整公开数据,较难完成该任务;㈡语料规范性本来就差,不差这一项,而词袋模型鲁棒性较强;㈢我们假设用字习惯和语料来源网站没有强相关性;㈣即使有强相关性,也主要干扰的是文本聚类,对文本分类的干扰较小,以上四个理由,而不作这方面工作。

经过以上处理后,将每个文档表示为一个字符串序列,每个元素为一个词,并将作者、流派及文档表示存储起来,每个作者的文章存为一个JSON文件,保存在\verb!./corpus!目录下。

本部分代码见于\verb!./preprocess/segmentation.py!文件。

\section{文本分类模型训练}




%\begin{figure}[htbp]
%	\centering
%	\includegraphics[width=.8\textwidth]{D:/pic.png}
%	\caption{這是標題}
%\end{figure}




\end{document}