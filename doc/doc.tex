\documentclass[12pt, a4paper, oneside]{ctexart}
\usepackage{amsmath, graphicx, geometry, hyperref}


% 导言区

\title{基于scikit-learn库的文本分类任务测试}
\author{叶蓁 211506011}
\date{}

\setCJKmainfont{I.Ming-8.00.ttf}
\setCJKsansfont{LXGWClearGothic-Regular.ttf}
\setCJKmonofont{SourceHanMono.ttc}

%\geometry{left=1.5cm, right=1.5cm, top=3cm, bottom=3cm}

\begin{document}

\maketitle

\section{任务简介}

本任务目的是进行一次基于传统机器学习的文本分类任务实践。

现今机器学习已经有大量很成熟的工具,不需要我们从头开始设计。总的来说,有低代码可视化平台和编程工具包两类工具可供我们使用,为了更加自由地处理数据和更加便捷地测试和输出测试结果,我们选择使用编程语言工具包来完成。具体来说,我们选择使用python语言的scikit-learn库来完成,因为scikit-learn库是当前最为常用的机器学习工具包,且提供的算法丰富,可供我们进行大量比较,API结构清晰,便于使用,还提供了便捷的测试结果统计和评估函数。

本任务选取的分类对象是明末公安派与清初桐城派的散文。在文学史上,桐城派很大程度是为反对公安派而产生的,两派散文有较大差异,应当能够被简单的基于机器学习的文本分类模型区分。两派的文章差异体现在用词、句法、篇章结构、思想内涵等多个方面,后两个部分一般来说由篇章分析和主题提取两类任务负责。本任务仅基于用词特点,建立词袋模型(Bag-of-words model)进行训练。

总的流程是,先对两派散文分别选取一定的范围,并在网络上爬取,存储下来。需要注意的是,不同网络资源的结构化程度不同,结构化较强的,已以篇为单位整理好,并施句读;结构化较弱的,可能不仅无句读,且连篇章界限亦无显性标识。对于后者,需要根据文本规律划分篇章。在可能的情况下,应尽量选取结构化程度更高的资料,以避免预处理中不必要的工作。

接下来,对文本进行预处理。具体来说,对于无句读的文档进行断句。尔后对每篇文档进行分词处理,将每篇文档表示为一个字符串序列。在此之后,根据字符串序列建立独热(one-hot)表示或TF-IDF加权表示,将每篇文档表示为一个向量。

将向量化的文档分别以互信息筛选和方差筛选两种方法降维,并以一定比例划分为训练集和测试集后,分别采取支持向量机(SVM)、对数几率回归(logistic regression)、朴素贝叶斯(naive Bayes)、K近邻(K-nearest neighbors)、决策树(decision tree)、随机森林(random forest)六种算法根据训练集进行建模,并应用于测试集。记录下测试集的预测结果与实际结果,计算每种情况的F值与AUC值,分析对于此任务,各算法的优劣。

\section{语料库制作}

任务的第一步是爬取网络资源制作语料库。对于公安派和桐城派,我们分别选取其最具代表性的公安三袁的作品集,袁宏道\href{《袁中郎全集》}{https://ctext.org/wiki.pl?if=gb&res=142162&remap=gb}、袁中道\href{《珂雪斋集》}{https://zh.wikisource.org/wiki/%E7%8F%82%E9%9B%AA%E9%BD%8B%E9%9B%86}、袁宗道\href{《白苏斋类集》}{https://zh.wikisource.org/zh-hant/%E7%99%BD%E8%98%87%E9%BD%8B%E9%A1%9E%E9%9B%86}和桐城三祖的作品集方苞\href{《方望溪先生全集》}{https://zh.wikisource.org/wiki/%E6%96%B9%E6%9C%9B%E6%BA%AA%E5%85%88%E7%94%9F%E5%85%A8%E9%9B%86_(%E5%9B%9B%E9%83%A8%E5%8F%A2%E5%88%8A%E6%9C%AC)}、刘大櫆\href{《海峰文集》}{https://ctext.org/wiki.pl?if=gb&res=110875}、姚鼐\href{《惜抱轩文集》}{https://zh.wikisource.org/wiki/%E6%83%9C%E6%8A%B1%E8%BB%92%E6%96%87%E9%9B%86},它们分别可以在维基文库和中国哲学书电子化计划两网站上看到,且每部集子都配有带链接的目录页,便于爬取。维基媒体的所有网站都是不设反爬措施的(实际上维基媒体鼓励合规的自由利用网站数据),中国哲学书电子化计划网站设有反扒措施,但并不严密,且主要是针对爬取网站公开书影的,对文本信息页面爬取管得不严,综上,我们不专门设计避免反扒措施的流程,不过伪装浏览器的请求头还是必要的。

此外,本任务仅是对散文流派的文本分类,其他文体不仅在用词上与散文有较大差异,在文学上也不被纳入桐城派、公安派的分类之中。以上作品集有的并非是散文类集性质,其中包含的其他文体因而是需要排除的。还有一些目录、序一类章节,也是需要排出的。这需要排除的内容被手动写进了代码。




%\begin{figure}[htbp]
%	\centering
%	\includegraphics[width=.8\textwidth]{D:/pic.png}
%	\caption{這是標題}
%\end{figure}




\end{document}